
%Document generated using DScaffolding from https://www.mindmeister.com/1276187433
\documentclass{article}
\usepackage[utf8]{inputenc}

\newcommand{\todo}[1] {\iffalse #1 \fi} %Use \todo{} command for bringing ideas back to the mind map

\title{WebAnnotationSPL}
\author{}

\begin{document}

\maketitle
      

\section{Introduction}

%Describe the practice in which the problem addressed appears

    
%Describe the practical problem addressed, its significance and its causes
Research has shown that a major problem within Practice for web annotation systems users is that Web annotation tools are develop from scratch \cite{Gayoso-Cabada2013}. This problem is of particular concern as it is now well established that it can lead to Limited interoperability \cite{Kalboussi2016}, The goldilock issue. There is a great variety of tools, making more difficult to choose the annotation system required for a certain scenario \cite{Kalboussi2016} \cite{CohenTOWARDSTHE}, Large Time to market \cite{CohenTOWARDSTHE} \cite{CohenTOWARDSTHE} and Large rate of annotation tools being unmaintained and likely, abandoned \cite{Neves2014}. Causes can be diverse: (1) Domain specificness. Annotations activities (annotation production, management and consumption) are conducted in different ways depending on the annotation purpose, the workflow domain, the annotated target, etc. \cite{Kalboussi2015} \cite{Kalboussi2016} \cite{Kalboussi2016} \cite{Ghadirian2018} and (2) Badly documented implementation or non-open source software \cite{Neves2019}. 
    
%Summarise existing research including knowledge gaps and give an account for similar and/or alternative solutions to the problem

    
%Formulate goals and present Kernel theories used as a basis for the artefact design
In this work, we address 1 main cause: No reusable components, libraries of annotation systems. To lessen this cause, we resort to Software Product Lines architecture facilitates code reuse and extension reducing costs and "One of the key factors for improving quality, productivity and consequently reducin g costs in software development is the adoption of software reuse - the process of creating software systems from existing software rat her than building them from scratch [Krueger, 1992].". 
    
%Describe the kind of artefact that is developed or evaluated
This article presents an artefact named SPL&Go. This artefact is a SPL&Go is a software product line which allows the generation of annotation tools with the required features in a certain annotation activity. 
    
%Formulate research questions
In summary, along Wieringa's template \cite{Wieringa2014}, this paper's design problem can be enunciated as follows: 
improve Web annotation tools are develop from scratch
by designing a(n) SPL&Go is a software product line which allows the generation of annotation tools with the required features in a certain annotation activity
that satisfies Modularity, Usability, Customisability, Accessibility, Maintainability, Flexibility, Interoperability and Developing a software product line where developers can take advantage of already implemented components, reuse and extend with further features
in order to help Academia: students, teachers and researchers achieve Conduct their annotation activities in the less time possible and Web annotator developers achieve Reduce the amount of resources (time, number of developers,...) required to develop an annotation systemReuse already-existing libraries, frameworks,... to annotate, store, consume, manage. In summary, don't reinvent the wheel. 
    
%Summarize the contributions and their significance
It is hoped that this research will contribute to a deeper understanding of Practice for web annotation systems users. We propose a solution aiming at complementing current approaches for solving Web annotation tools are develop from scratch. Consequently, this study can be classified as an improvement along Gregor and Hevner’s DSR knowledge contribution framework \cite{Gregor2013}.
      
%Overview of the research strategies and methods used
This article has followed a Design Science Research approach.

%Describe the structure of the paper
The remainder of the paper is structured as follows: 

%Optional - illustrate the relevance and significance of the problem with an example
    
      
\bibliographystyle{unsrt}
\bibliography{references}

\end{document}
    